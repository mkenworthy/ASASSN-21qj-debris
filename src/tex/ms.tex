%\documentclass[referee]{aa} % for a referee version
%\documentclass[onecolumn]{aa} % for a paper on 1 column  
%\documentclass[longauth]{aa} % for the long lists of affiliations 
%\documentclass[letter]{aa} % for the letters 
\documentclass{aa}

\usepackage{txfonts}
\usepackage{natbib}

\usepackage{graphicx}

\usepackage{color}
\usepackage{hyperref}
\hypersetup{colorlinks=true,allcolors=[rgb]{0,0,0.8}}


\usepackage{showyourwork}

% the three lines suppress the hyperref 'link empty' warnings
% explanation at: https://tex.stackexchange.com/questions/345764/journal-class-shows-package-hyperref-warning-suppressing-link-with-empty-targe
\makeatletter
\renewcommand*\aa@pageof{, page \thepage{} of \pageref*{LastPage}}
\makeatother

\usepackage{xspace}

\newcommand{\ktwo}{\textit{K2}}
\newcommand{\kms}{km~s$^{-1}$\xspace}
\newcommand{\ms}{m~s$^{-1}$}
\newcommand{\gcc}{g~cm$^{-3}$}
\newcommand{\masyr}{mas~yr$^{-1}$}
\newcommand{\err}{\textit{$\pm$}}
\newcommand{\teff}{$T_\mathrm{eff}$}
\newcommand{\msun}{$M_\odot$}
\newcommand{\rsun}{$R_\odot$}
\newcommand{\lsun}{$L_\odot$}
\newcommand{\rhosun}{$\rho_\odot$}
\newcommand{\mstar}{$M_*$}
\newcommand{\rstar}{$R_*$}
\newcommand{\lstar}{$L_*$}
\newcommand{\rearth}{$R_\oplus$}
\newcommand{\vrad}{$v_{R}$}
\newcommand{\pmra}{$\mu_{\alpha}$}
\newcommand{\pmdec}{$\mu_{\delta}$}

\newcommand{\rhostar}{$\rho_*$}
\newcommand{\mjup}{$M_\mathrm{Jup}$}
\newcommand{\galex}{\textit{GALEX}}
\newcommand{\gaia}{\textit{Gaia}}
\newcommand{\kepler}{\textit{Kepler}}
\newcommand{\spitzer}{\textit{Spitzer}}
\newcommand{\ktwosc}{\textsc{k2sc}}
\newcommand{\ktwosff}{\textsc{k2sff}}
\newcommand{\hipparcos}{\textit{Hipparcos}}
\newcommand{\tess}{\textit{TESS}}
\newcommand{\emcee}{\textsc{emcee}}
\newcommand{\python}{\textsc{python}}


\begin{document} 

   \title{The eclipse of ASASSN-21qj}

   \author{M. Kenworthy
          \inst{1}
          \and
          Arttu Sainio
          \inst{2}
          \and
          Eric E. Mamajek
          \inst{3}
          \and
          Joeseph Masiero
          \inst{4}
          \and 
          Amy Mainzer
          \inst{4}
          \and
          Joeseph (Davy) Kirkpatrick
          \inst{4}
          \and 
          Grant M. Kennedy
          \and
          Ludmila Carone
          \and
          NEOWISE authorship list
          \inst{4}
          \and
          AAVSO observers
          \and
          St\'{e}phane Charbonnel
          \and
        Olivier Garde
        \and
        Pascal Le D\^{u}
        \and
        Lionel Mulato
        \and
        Thomas Petit
          }

   \institute{Leiden Observatory, University of Leiden,
   PO Box 9513, 2300 RA Leiden, The Netherlands\\
   \email{kenworthy@strw.leidenuniv.nl}
         \and
             Arttu's address\\
    \and
    JPL
    \and
    Caltech/IPAC, 1200 E California Blvd, Mail Code 100-22, Pasadena, CA 91125, USA
    \and
    NEOWISE
    \and
    AAVSO people
    \and
    Space Research Institute, Austrian Academy of Sciences, Schmiedlstrasse 6, A-8042 Graz, Austria
}
   \date{Received XXXX; accepted XXXX}

% \abstract{}{}{}{}{} 
% 5 {} token are mandatory
 
  \abstract
  % context heading (optional)
  % {} leave it empty if necessary  
   {Collisions occur between planetessimals that generate debris disks through collisional cascades.}
  % aims heading (mandatory)
   {We analyze the dust and size distribution of the eclipse seen towards ASASSN-21qj.}
  % methods heading (mandatory)
   {Fit the light curve from three different colours to determine the particle size and distribution.}
  % results heading (mandatory)
   {The eclipse is coloured, indicating dust.
   %
   The dust has a lower limit mass of XXXX Earths, eclipse has a duration of XXXX days.}
  % conclusions heading (optional), leave it empty if necessary 
   {}

   \keywords{giant planet formation --
                $\kappa$-mechanism --
                stability of gas spheres
               }

   \maketitle
%
%-------------------------------------------------------------------

   \section{Introduction}

Terrestrial planets are thought to be built up by the quasi-periodic accretion of planetary embryos that generate a significant amount of ejected material.
%
The Earth's moon is believed to have formed from the resulting aftermath of a collision in the early Solar system.
%
Sudden increases of infrared flux from systems known to host debris disks indicate that this is a stochastic process that can occur on timescales of a few months or less.
%
Models of these impacts and the subsequent evolution of the dust clouds have been modeled \citep{Jackson12,Jackson14} and have been seen ar IR wavelengths \citep{Su19,Su22}

   The star (Gaia EDR3 source 5539970601632026752 at RA=08:15:23.2996, DEC=-38:59:23.304, $d\sim 556$ pc, G=13.4 mag, BP-RP=0.8 mag) underwent a sudden dimming event in December 2021, which was announced by \citet{RizzoSmith21} and assigned the identifier ASASSN-21qj.
   %
   The star has subsequently maintained rapidly fluctuating photometry through to August 2022 \citep{RizzoSmith22} and has been monitored intensively by the AAVSO observers and the LCOGT network of telescopes.
   %
   It had previously shown no significant stellar variation in the optical bands for 2300 days as reported in \citet{RizzoSmith21}.
   %
   Searches through other photometric archives showed no other significant changes in the optical bands before this epoch.
   %
   The wide field infrared satellite WISE has photometric imaging from 3.8 microns (band W1) through to 25 microns (band W4), and the NEOWISE survey has photometry for bands W1 and W2 for this star.
   %
   This star showed a significant brightening of 0.7 magnitudes in W1 and 0.8 magnitudes in W2 between two observing epochs (65000 MJD and 65120 MJD), and the IR color of the star had changed from W1-W2=0.0 to 1.2, but no significant changes in flux in the optical bands were seen during this time.
   %
   Some 900 days later, the dimming was seen in the optical, and the absorption is larger at shorter optical wavelengths.
   %
   These observations are all consistent with an event that generated a significant amount of sub-micron dust which subsequently started to transit the stellar disk.

   We hypothesise that there was a collision between one or more rocky bodies in the system which generated a significant amount of dust, which has then subsequently begun to transit the star.
   %
   This is consistent with a late type impact between a planet and large asteroid, similar to the one that generated the Earth/Moon system.

   The structure of our paper is as follows: the analysis of the star is given in Section~\ref{sec:star}, the observations are detailed in Section~\ref{sec:obs}, and we make an estimate of the physical parameters of the hypothesised dust cloud in Section~\ref{sec:dustcloud}.
   %
   We then place this model in the context of planet formation in Section~\ref{sec:discussion} and summarise the paper in Section~\ref{sec:conclusion}.


Statistical distributions of ejecta masses from collision in \citet{Crespi21}.

Planetary embryo collisions and wiggly disks.... \citet{Watt21}.

\section{Properties of the star}\label{sec:star}

The properties of ASASSN-21qj (Gaia EDR3 source 5539970601632026752) are listed in Table~\ref{tab:Stellarprop}.

\begin{table}
    \centering
    \caption{ Properties of ASASSN-21qj}
    \begin{tabular}{@{}lcc@{}}
    \hline\hline
Property                               & Value                     & Ref.  \\
        \hline
         $\alpha_{ICRS}$, J2000 {[}hh mm ss{]}  & 08:15:23.30              & 1     \\
         $\delta_{ICRS}$, J2000 {[}dd mm ss{]}  & -38:59:23.3              & 1     \\
         $\mu_{\alpha}$ {[}mas yr$^{-1}${]}     & $-9.692\pm0.012$          & 1     \\
         $\mu_{\delta}$ {[}mas yr$^{-1}${]}     & $7.349\pm0.012$          & 1     \\
         $\varpi$ {[}mas{]}                     & $1.763\pm0.011$         & 1     \\
         Distance {[}pc{]}                      & $XXXX^{+6.7}_{-5.9}$     & 2     \\ 
        \hline
         $G$ {[}mag{]}                          & $13.371\pm XXXX$        & 1     \\
         $G_{BP}-G_{RP}$ {[}mag{]}              & $0.815\pm XXXX$        & 1     \\
         $G_{BP}$ {[}mag{]}                     & $13.697\pm XXXXX$        & 1     \\
         $G_{RP}$ {[}mag{]}                     & $12.882\pm XXXXX$        & 1     \\
%         $J$ {[}mag{]}                          & $12.897\pm0.026$          & 3     \\
%         $H$ {[}mag{]}                          & $12.431\pm0.024$          & 3     \\
%         $K$ {[}mag{]}                          & $12.321\pm0.024$          & 3     \\
%         $u$ (AB) {[}mag{]}                     & $17.63\pm0.02$            & 4     \\
%         $g$ (AB) {[}mag{]}                     & $16.096\pm0.008$          & 4     \\
%         $r$ (AB) {[}mag{]}                     & $16.69\pm0.02$            & 4     \\
%         $i$ (AB) {[}mag{]}                     & $14.123\pm0.005$          & 4     \\
%         $z$ (AB) {[}mag{]}                     & $14.832\pm0.011$          & 4     \\
        \hline
         $R_*$ [\rsun{}]                        & $1.009\pm0.030$           & 5     \\
%         $M_*$ [$M_{\odot}$]                    & $0.85\pm0.02$             & 5     \\
         {[}Fe/H{]} [dex]                       & $0.0\pm0.23$               & 5     \\
         log\,$g$ [log$_{10}$ cm\,s$^{-2}$]     & $4.5\pm0.25$               & 5     \\
         \teff{} [K]                            & $5900\pm74$               & 5     \\
%         $f_{bol}$ [10$^{-11}$ erg s$^{-1}$]    & $5.288\pm0.130$           & 5     \\
%         $m_{bol}$ [mag]                        & $14.194\pm0.027$          & 5     \\
         $m_{bol}$ [mag]                        & $13.39\pm0.02$           & 5     \\
%         $L_{bol}$ [$L_{\odot}$]                & $0.046\pm0.013$         & 5     \\
         log($L/L_{bol}$) [dex]                 & $0.046\pm0.013$        & 5     \\
         
        \hline
    \end{tabular}
    \tablefoot{References:
    (1) Gaia EDR3 \citep{Brown21},
    (2) \citet{BailerJones21},
    (3) 2MASS \citep{Cutri03},
    (4) SDSS DR8,
    (5) this work (EEM fits)
    }
    \label{tab:Stellarprop}
\end{table}

% Including IR WISE photometry
%was forcing it to lower metallicity ([Fe/H] ~ -0.5-1), cooler (~Sun) fits. Fits also favor low reddening values ($A_v\sim 0.05$). 
%Similar to Alpha Cen A, but ~solar metallicity seems OK now. Almost exactly 1 Rsun! 

%Best parameters as of 1/2/2022 (see below from VOSA fit) 
%Teff = 5900 +- 74 K
%log(L/Lsun) = 0.046+-0.013 
%Av = 0.05+-0.03
%mbol = 13.39+-0.02 (apparent)
%logg = 4.5+-0.25
%[M/H] = 0.0+-0.23 
%Rad = 1.009 +- 0.030 Rsun

Test


\begin{figure}
   \centering
   \includegraphics[width=\hsize]{figures/asassn-21qj-gmk-sed-fit.png}
      \caption{VOSA fit to photometry of ASAS-SN J0815 by gmk}
         \label{fig:sed}
\end{figure}


\begin{figure}
   \begin{centering}
   \includegraphics[width=\hsize]{figures/filter_curves.pdf}
      \caption{Filter curves for all telescopes and filters used to observe ASASSN-21qj.}
      \label{fig:allfilters}
      \script{plot_filter_curves.py}
      \end{centering}
\end{figure}



\begin{figure}
   \begin{centering}
   \includegraphics[width=\hsize]{figures/mie_single.pdf}
      \caption{Mie test.}
      \label{fig:mietest}
      \script{plot_mie_single.py}
      \end{centering}
\end{figure}




\begin{figure*}
   \begin{centering}
   \includegraphics[width=\textwidth]{figures/eclipse_overview2.pdf}
      \caption{Photometry from the optical bands of the eclipse.
      %
The different telescopes and filters are indicated in the legend.
%
Each light curve is offset vertically by 0.8.
              }
        \label{fig:eclipse_overview}
        \script{plot_eclipse_overview2.py}
    \end{centering}
\end{figure*}



\subsection{Possible multiplicity in the  system}

ASASSN-21qj (Gaia DR3 5539970601632026752 = 2MASS J08152329-3859234) has a neighbor (Gaia DR3 5539970597334497024= 2MASS J08152298-3859244) a few arcseconds away for which we checked to see if it could plausibly be a physical companion.
%
Based on the Gaia EDR3 mean ICRS position for epoch 2016.0, the visual companion lies at a separation $\rho$ = $3738.243\pm0.062$ mas and at position angle $\theta$ = $249^{\circ}.977$.
%
Their parallaxes ($\varpi$ = $1.7631\pm0.0112$ mas vs. $1.4711\pm0.0523$ mas) differ by 5.5$\sigma$ and proper motions ($\mu_{\alpha} = -9.692\pm0.012$, $\mu_{\delta} = 7.349\pm0.012$ \masyr\, vs. $\mu_{\alpha} = -0.114\pm0.055$, $\mu_{\delta} = 6.419\pm0.053$ \masyr) differ by a factor of 2.
%
If the two stars are at the parallactic distance of ASASSN-21qj
\citep[$d$ = 552.4 pc;][]{BailerJones21}, then their difference in proper motion ($\Delta\mu$ = $9.623\pm0.056$ \masyr) translates to a difference in tangential velocity of $\Delta V_{tan}$ = $25.37\pm0.21$ \kms.
%
The velocity difference is considerable given the projected separation of 2079\,au.
%
Using Kepler's 3rd law, and assuming the observed separation corresponded
to the semi-major axis, and the observed $\Delta V_{tan}$ corresponded to the full orbital velocity, these quantities would predict a minimum system mass of $>$1509 Msun.
%
Based on the implausible estimated dynamical mass inferred from the observed separation and difference in tangential velocities, we conclude that the visual companion Gaia DR3 5539970597334497024 (2MASS J08152298-3859244) is an interloper
unrelated to ASASSN-21qj.

\section{Observations}\label{sec:obs}
An overview of all the photometry is presented in Figure~\ref{fig:allphot}.




\subsection{ASAS-SN photometry}

The beginning of the eclipse was identified in \citet{RizzoSmith21} through the ASAS-SN survey.
%
\subsection{ASAS}

The All Sky Automated Survey \citep[ASAS; ][]{pojmanski_all_1997, asas_2005, asas_2018} is a survey consisting of two observing stations - one in Las Campanas, Chile and the other on Maui, Hawaii. 
%
Each observatory is equipped with two CCD cameras using V and I filters and commercial f $ = 200$ mm, D $= 100$ mm lenses, although both larger (D $= 250$ mm) and smaller (50-72 mm) lenses were used at earlier times.
%
The majority of the data are taken with a pixel scale of $\approx$ 15\arcsec{}.
%
ASAS splits the sky into 709 partially overlapping (9\degr{} $\times$ 9\degr{} fields, taking on average 150 3-minute exposures per night, leading to a variable cadence of 0.3-2 frames per night.
%
Depending on the equipment used and the mode of operation, the ASAS limiting magnitude varied between 13.5 and 15.5 mag in V, and the saturation limit was 5.5 to 7.5 mag. 
%
Precision is around 0.01-0.02 mag for bright stars and below 0.3 mag for the fainter ones. 
%
ASAS photometry is calibrated against the Tycho catalog, and its accuracy is limited to 0.05 mag for bright, non-blended stars.

\subsection{ASAS-SN}

The All Sky Automated Survey for Supernovae \citep[ASAS-SN; ][]{shappee_man_2014,kochanek_all-sky_2017} consists of six stations around the globe, with each station hosting four telescopes with a shared mount.
%
The telescopes consist of a 14-cm aperture telephoto lens with a field of view of approximately 4.5\degr{}$\times$4.5\degr{} and an 8.0\arcsec{} pixel scale.
% 
Two of the original stations (one in Hawaii and one in Chile) are fitted with $V$ band filters, whereas the other stations (Chile, Texas, South Africa and China) are fitted with $g$ band filters.
%
ASAS-SN observes the whole sky every night with a limiting magnitude of about 17 mag in the $V$ and $g$ bands.
%

\subsection{TESS}

The Transiting Exoplanet Survey Satellite \citep[TESS; ][]{2015JATIS...1a4003R} is a satellite designed to survey for transiting exoplanets among the brightest and nearest stars over most of the sky.
%
The TESS satellite orbits the Earth every 13.7 days on a highly elliptical orbit, scanning a sector of the sky spanning 24\degr $\times$ 96\degr\ for a total of two orbits, before moving on to the next sector. 
%
It captures images at a 2 second (used for guiding), 20 seconds (for 1 000 bright asteroseismology targets), 120 seconds (for 200 000 stars that are likely planet hosts) and 30 minutes (full frame image) cadences.
%
The instrument consists of 4 CCDs each with a field of view of 24\degr$\times$24\degr, with a wide band-pass filter from 600-1000 nm (similar to the $I_C$ band) and has a limiting magnitude of about 14-15 mag ($I_C$).
%
The data was extracted from the TESS archive using the {\tt eleanor} package \citep{Feinstein19} which corrected for known systematics in the cameras and telescope.

\subsection{ROAD Photometry}



\subsection{NEOWISE photometry}

The NEOWISE photometry is presented with the ASASSN $q$ light curve in Figure~\ref{fig:wisephot}.

\begin{figure*}
\begin{centering}
\includegraphics[width=\textwidth]{figures/all_photometry.png}
\caption{NEOWISE $W1$ and $W2$ photometry of the star, with the WISE color in the lowest panel.
%
The $NEOWISE$ color changes from colourless to very red, which fades back towards colourless over $\sim 500$ days.
}
\label{fig:wisephot}
\script{plot_all_photometry.py}
\end{centering}
\end{figure*}


\begin{figure*}
\begin{centering}
\includegraphics[width=\textwidth]{figures/stellar_lomb_scargle.pdf}
\caption{ASASSN photometry of ASASSN-21qj and the Lomb Scargle periodograms of the photometry in and out of the eclipse.
%
The blue and orange shaded regions in the top panel indicate the range of epochs put into the Lomb Scargle periodogram.
%
Middle panel: The periodograms over a range of 0 to 150 days.
%
Lower panel: The periodogram of the star outside of the eclipse over a range of 0 to 50 days.
}
\label{fig:starlombscargle}
\script{calc_stellar_lomb_scargle.py}
\end{centering}
\end{figure*}



NEOWISE....

\variable{output/collision_epoch_text.txt}


\subsection{LCOGT photometry}

\subsection{ATLAS}

ATLAS photometry was obtained from their database.



ATLAS is a project that searches for near earth asteroids down to a magnitude of 19 
\citep{Tonry18}.
%
Two filters were obtained, the `o' and `c' filters respectively.
%
Photometry consists of two to four photometric points observed each night when conditions permitted.
%
Photometry with large errors was rejected in a first pass.
%
The remaining observations during a night were averaged and an error based on the r.m.s. of these nightly points was calaulated.
%
The photometry covers the time period where the collision event occurred. 

\subsection{2SPOT spectroscopy}

For the observations and setup : 
Ritchey-Chretien telescope of 12 inch in diameter (305mm) open at $f/5$ (with focal reducer CCD 67 astrophysics).
%
Equatorial mount GM 3000 HPS from 10 micron (https://www.10micron.com/en/product/gm3000-hps/)

The spectrograph is a Spectrograph Alpy 600 $R=570$ with a $23\mu m$ wide slit (https://www.shelyak.com/produit/alpy-600/?lang=en) and an ATIK 414ex camera. (https://www.atik-cameras.com/product/atik-414ex/)

The observation site is in Chile at Deep Sky Chile (https://www.deepskychile.com/en/) 

Lon : 70°W 51’ 11,86’'
Lat : 30°S 31’ 34,71 »
Alt : 1700m AMSL

For this spectrum : 3 exposures of 1200s each in bin 1x1 taken in automatic mode with Prism V11 Software (https://www.prism-astro.com/) , spectrum process with ISIS software (http://www.astrosurf.com/buil/isis-software.html)
Date : 07/09/2022 at 8h34’52’ UTC (middle time exposure) JJ Date : 2459829,8785

And for more informations about 2SPOT, you wil find many informations in our website : https://2spot.org/EN/
and also information about the team : https://2spot.org/EN/equipe.php


\begin{figure*}
   \begin{centering}
   \includegraphics[width=\textwidth]{figures/2spot_spectrum.pdf}
      \caption{2plot spectrum showing features of a G type star.
      %
              }
              \label{fig:2spotspectrum}
              \script{plot_2spot_spectrum.py}
              \end{centering}
       \end{figure*}




\section{Analysis}\label{sec:dustcloud}





\subsection{Dust properties from the colors in the optical}



\begin{figure*}
   \begin{centering}
   \includegraphics[width=\textwidth]{figures/scale_combined_photometry.pdf}
      \caption{Photometry from the optical bands of the eclipse scaled arbitrarily so as to combine the light curves into a ``gray'' light curve.
      %
      The axis is inverted to show Absorption.
              }
              \label{fig:allphot}
              \script{plot_scale_combined_photometry.py}
              \end{centering}
       \end{figure*}



\section{Discussion}\label{sec:discussion}

We have two separate phenomena that occur within 1000 days of each other.
%
Firstly, NEOWISE W1 and W2 photometry towards the star shows an increase in flux of a factor of two in both wavebands, with the longer wavelength showing a larger increase.
%
Observations from the optical bands have a much higher cadence, but do not show any corresponding increase in flux.
%
This infrared only increase is consistent with a dust generating event, such as the collision of a planetoid with another planetoid or a terrestrial planet (REF REF).
%
The resultant dust cloud has a considerably larger surface area than the progenitors, and this dust cloud is then heated by the flux of the star and rapidly reaches thermal equilibrium.
%
The temperature of the dust cloud can be estimated from spectral energy distribution measured towards the stellar system.

About 1000 days after the IR flux increase, the stellar flux begins to decrease.
%
The absorption is larger at bluer wavelengths and is consistent with sub-micron sized dust absorption (REF REF).

\subsection{Infrared increase}

This increase occurred between MJD \variable{output/t_before.txt} and MJD \variable{output/t_after.txt} dates, a period of approximately \variable{output/t_duration.txt} days.
%
Subsequent NEOWISE measurements show an approximately constant excess flux for XXXX days, followed by a decrease from XXXX to XXX.

We fit a blackbody SED to the excess flux in the two NEOWISE bands for each of the epochs.
%
Assuming grey emission (where $\upsilon(\lambda)\approx \upsilon$) we obtain an effective temperature of $_{IR} = 900\pm XXX$K, and $F_{IR}/F_{star}=0.03$.
%
The sudden onset of the IR flux within 155.21 days followed by a constant level of flux implies that the dust cloud is optically thin by the time of the first measurement, and therefore has had to expand such that 0.03 of the solid angle around the star was subtended by dust and reradiated in the IR.
%
An extimate for the surface area $A_{IR}$ of the dust can be estimated from $A_{IR}/(4\pi r_{IR}^2) = 0.03$ which leads to:

$$$A_{IR}=0.38 a_{IR}^2$$ 

where $a_{IR}$ is the distance of the dust from the star.

If we assume that the dust cloud expanded from a point source out into a star-facing circular optically thin disk with area $A_{IR}$ and radius $r_{IR}$ in a time $t_{IR}$ (where this time has an upper limit of 6 months corresponding to the duration between the NEOWISE epochs), then we can estimate a linear velocity $v_{IR} t_{IR} = r_{IR}$ for the expansion of the cloud of:

$$v_{IR} = \frac{r_{IR}}{t_{IR}} = \frac{0.35 a_{IR}}{t_{IR}} = $$
%# A_{IR} / 4 pi (a_{IR})2 =  pi r_{IR}^2 / 4 pi a2 = 0.03
% r2 / (4a2) = 0.03
% $r_{IR} = 2a_{IR} sqrt(0.03)$





\subsection{Optical absorption}

Hansen collision of terrestrial planet with scattered moon \citep{2022MNRAS.tmp.2636H}

Cound be a ring system

Could be a planetessimal system evolving

Alternatives?

\textcolor{magenta}{Probing disintegrating planetary material has been proven to be a very useful method to access the composition of building blocks of planets outside of the Solar System.
%
E.g. the disintegration of a gas giant around a white dwarf was used to infer its composition \citep{Gaensike2019}.
%
White dwarf star pollution has also yielded insights about refractory (Fe/Mg/Ca) element content of rocky  material around such stars \citep{Turner2020,Putirka2021,Blouin2020}.
%
See also \cite{Veras2021} for a review.}

\textcolor{magenta}{Planets and asteroids falling into white dwarfs, however, haven been probably heavily processed during the late stellar evolution.
%
The unusually warm (how much?) debris disk passing in front of a young star presented in this work, on the other hand, may allow to probe the interior of planetesimals in the early stages of planet formation.
%
The eclipse is expected to last for xxx days, allowing to perform further spectroscopic analysis of the dust in this system.}

%------------------------
\section{Conclusions}\label{sec:conclusion}

   \begin{enumerate}
      \item There was a collision between planetoids towards ASASSN-21qj which generated a debris cloud.
      %
      \item The cloud moved in front of the star, and we have a fresh measure of the debris from a collision.
      %
     \item \textcolor{magenta}{Probing dust of material in the early stages of planet formation is complementary to studies of white dwarf polluters.
    %
    The latter represent planetary material after the end of the main sequence of the host star. }
   \end{enumerate}

\begin{acknowledgements}

This research has used the SIMBAD database, operated at CDS, Strasbourg, France \citep{wenger2000}.
%
This work has used data from the European Space Agency (ESA) mission {\it Gaia} (\url{https://www.cosmos.esa.int/gaia}), processed by the {\it Gaia} Data Processing and Analysis Consortium (DPAC, \url{https://www.cosmos.esa.int/web/gaia/dpac/consortium}).
%
Funding for the DPAC has been provided by national institutions, in particular the institutions participating in the {\it Gaia} Multilateral Agreement.
%
To achieve the scientific results presented in this article we made use of the \emph{Python} programming language\footnote{Python Software Foundation, \url{https://www.python.org/}}, especially the \emph{SciPy} \citep{virtanen2020}, \emph{NumPy} \citep{numpy}, \emph{Matplotlib} \citep{Matplotlib}, \emph{emcee} \citep{foreman-mackey2013}, and \emph{astropy} \citep{astropy_1,astropy_2} packages.
%
We acknowledge with thanks the variable star observations from the AAVSO International Database contributed by observers worldwide and used in this research.
%
We thank the Las Cumbres Observatory and its staff for its continuing support of the ASAS-SN project, and the Ohio State University College of Arts and Sciences Technology Services for helping us set up and maintain the ASAS-SN variable stars and photometry databases.
%
ASAS-SN is supported by the Gordon and Betty Moore Foundation through grant GBMF5490 to the Ohio State University and NSF grant AST-1515927.
%
Development of ASAS-SN has been supported by NSF grant AST-0908816, the Mt. Cuba Astronomical Foundation, the Center for Cosmology and AstroParticle Physics at the Ohio State University, the Chinese Academy of Sciences South America Center for Astronomy (CASSACA), the Villum Foundation, and George Skestos.
%
Early work on KELT-North was supported by NASA Grant NNG04GO70G.
%
Work on KELT-North was partially supported by NSF CAREER Grant AST-1056524 to S. Gaudi.
%
Work on KELT-North received support from the Vanderbilt Office of the Provost through the Vanderbilt Initiative in Data-intensive Astrophysics.
%
Part of this research was carried out in part at the Jet Propulsion Laboratory, California Institute of Technology, under a contract with the National Aeronautics and Space Administration (80NM0018D0004).
%
This publication makes use of VOSA, developed under the Spanish Virtual Observatory project supported by the Spanish MINECO through grant AyA2017-84089.
%
This work has made use of data from the Asteroid Terrestrial-impact Last Alert System (ATLAS) project.
%
The Asteroid Terrestrial-impact Last Alert System (ATLAS) project is primarily funded to search for near earth asteroids through NASA grants NN12AR55G, 80NSSC18K0284, and 80NSSC18K1575; byproducts of the NEO search include images and catalogs from the survey area.
%
This work was partially funded by Kepler/K2 grant J1944/80NSSC19K0112 and HST GO-15889, and STFC grants ST/T000198/1 and ST/S006109/1.
%
The ATLAS science products have been made possible through the contributions of the University of Hawaii Institute for Astronomy, the Queen’s University Belfast, the Space Telescope Science Institute, the South African Astronomical Observatory, and The Millennium Institute of Astrophysics (MAS), Chile.

\end{acknowledgements}

\bibliographystyle{aa}
\bibliography{bib}

\end{document}
